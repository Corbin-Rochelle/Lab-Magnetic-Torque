\documentclass[aps,prl,10pt,twocolumn,floatfix]{revtex4-2}
\usepackage{graphicx}
\usepackage{dcolumn}
\newcolumntype{d}{D{.}{.}{-1}}
\usepackage{url}
\usepackage{physics}

\bibliographystyle{apsrev4-2}


\begin{document}

%\begin{abstract}

%\end{abstract}


\title{Experiments Involving Magnetic Torque}
\author{C. T. Rochelle}
\email{ctr233@msstate.edu}
\author{K. J. Grimes}
\email{Katys@Email.edu}
\date{\today}
\affiliation{Department of Physics and Astronomy\\Mississippi State University\\Mississippi State, MS 39762-5167}
\date{\today}

\maketitle

\section{Introduction}\label{Intro}
% History of Magnetism
In the past, magnetism was shrouded in mystery. 
Unlike experiments involving classical mechanics, where objects interact with each other through visible and easily measurable interactions (i.e. collisions, steam, etc)\footnote{with a few notable exceptions like gravity}, experiments trying to understand magnetic phenomena were measuring and trying to describe events that could not be seen or felt.
Because of this, electro-magnetic theory came about much later and more slowly than classical theories. 

While the existence of magnets in compass needles and mysterious lodestones were known about by the Chinese and later European civilizations for thousands of years, it was only up until the turn of the 1st century CE that they were used for practical purposes, tracking shipping navigation\cite{History}.
It was only until the 16 and 1700s when the first explanations and experiments of magnetism had begun. 
Later on in the 20th century, non ferro-magnets began to be created and since magnetic theory was more advanced, they began to be harnessed in devices that we use every day. 
 
% History of Magnetic Torque 



% History of Hemholtz Coils


\section{Theory}\label{Theory}
% Derivation of Common Formulas Used in Experiments

% Theory of Hemholtz Coils

\section{Experiment}
% General Overview of All Experiments
% The Device
The magnetic torque instrument, pictured in Figure \ref{device}, consists of mounted Hemholtz coils with a brass air bearing holding a snooker ball in the middle of the field. 
The mounted coils are connected to a control box, pictured in Figure \ref{controlBox}, and to an outside air source to allow the snooker ball to be suspended in the bearing. 
The control box controls the current being fed to the coils, a controllable flashing LED mounted the coils, the direction of the field, and if the field has a gradient. 
The snooker ball, pictured in Figure \ref{snooker}, 

% Experiment 1

% Experiment 2

% Experiment 3

% Hall Effect

% Moment with Spring

% Balancing the Mass

\section{Data Analysis}
% Explain the Results 


\section{Conclusion}
% Conclusion 
% Talk about all values and if they agree with each other or not 




%\begin{thebibliography}{9}
%\bibitem{NIST} \textit{speed of light in vacuum} (National Institute of Standards and Technology, May 4, 2022).
%\bibitem{History} \textit{A Brief History} {https://www.magnetic-magnets.co.nz/technical/A-Brief-History.html}
%\end{thebibliography}

\end{document}
